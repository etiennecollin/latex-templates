%%%%%%%%%%%%%%%%%%%%%%%%%%%%%%%%%%%%%%%%%%%%%%%%%%%%%%%%%%%%%%%%%%%%%%%%%%%%%%%%%%%%%%%%%%%%%%%%%%%%%%%%%%%%%%%%%%%%%%%%
% Start of Environment Setup %%%%%%%%%%%%%%%%%%%%%%%%%%%%%%%%%%%%%%%%%%%%%%%%%%%%%%%%%%%%%%%%%%%%%%%%%%%%%%%%%%%%%%%%%%%
%%%%%%%%%%%%%%%%%%%%%%%%%%%%%%%%%%%%%%%%%%%%%%%%%%%%%%%%%%%%%%%%%%%%%%%%%%%%%%%%%%%%%%%%%%%%%%%%%%%%%%%%%%%%%%%%%%%%%%%%

\addbibresource{\latexTemplatesPath/bibliography.bib}    % Imports bibliography file
\graphicspath{{images/}} % Set images path

%%%%%%%%%%%%%%%%%%%%%%%%%%%%%%%%%%%%%%%%%%%%%%%%%%%%%%%%%%%%%%%%%%%%%%%%%%%%%%%%%%%%%%%%%%%%%%%%%%%%%%%%%%%%%%%%%%%%%%%%
% Maths %%%%%%%%%%%%%%%%%%%%%%%%%%%%%%%%%%%%%%%%%%%%%%%%%%%%%%%%%%%%%%%%%%%%%%%%%%%%%%%%%%%%%%%%%%%%%%%%%%%%%%%%%%%%%%%%
%%%%%%%%%%%%%%%%%%%%%%%%%%%%%%%%%%%%%%%%%%%%%%%%%%%%%%%%%%%%%%%%%%%%%%%%%%%%%%%%%%%%%%%%%%%%%%%%%%%%%%%%%%%%%%%%%%%%%%%%

% Math operators
\DeclareMathOperator{\im}{Im}


% Useful shortcuts
\newcommand{\B}{\ensuremath{\mathbb{B}}}
\newcommand{\N}{\ensuremath{\mathbb{N}}}
\newcommand{\R}{\ensuremath{\mathbb{R}}}
\newcommand{\Z}{\ensuremath{\mathbb{Z}}}
\renewcommand{\o}{\ensuremath{\emptyset}}
\renewcommand{\O}{\ensuremath{\mathcal{O}}}
\newcommand{\Q}{\ensuremath{\mathbb{Q}}}
\newcommand{\C}{\ensuremath{\mathbb{C}}}

% Distributions
\newcommand{\ber}{\ensuremath{\mathcal{B}}\text{er}}
\newcommand{\norm}{\ensuremath{\mathcal{N}}}
\newcommand{\expo}{\ensuremath{\mathcal{E}}\text{xp}}
\newcommand{\uni}{\ensuremath{\mathcal{U}}}

\newcommand{\alg}[1]{\textsc{\bfseries\footnotesize #1}}
\newcommand{\deriv}[2][]{\mathop{}\!\frac{\mathrm{d}^{#1}}{\mathrm{d}#2^{#1}}}      % For derivatives
\newcommand{\pderiv}[3][]{\mathop{}\!\dfrac{\partial^{#1}#2}{\partial#3^{#1}}}      % For partial derivatives
\newcommand{\pdiff}[2][]{\mathop{}\!\dfrac{\partial^{#1}}{\partial#2^{#1}}}         % For partial derivatives
\newcommand{\diff}[2][]{\mathop{}\!\mathrm{d^{#1}}#2}                               % Integral dx
\newcommand*{\widefbox}[1]{\fbox{\hspace{2em}#1\hspace{2em}}}                       % For boxed align envs
\newcommand*{\approxequiv}{\mathrel{\vcenter{\offinterlineskip\hbox{$\sim$}\vskip-.35ex\hbox{$\sim$}\vskip-.35ex\hbox{$\sim$}}}}    % For an approximately equivalent symbol
\newcommand{\mat}[1]{\bm{#1}}                                                       % ISO complying matrix variable font
\newcommand*{\carry}[1][1]{\overset{#1}}                                            % For carrying over a number

% SIunitx
% Sets \per to display as frac
\sisetup{per-mode=symbol-or-fraction}
\DeclareSIUnit\Molar{\mole\per\liter}
\DeclareSIUnit\molar{\textsc{M}}
\DeclareSIUnit\clight{\text{\ensuremath{c}}_{0}}

%%%%%%%%%%%%%%%%%%%%%%%%%%%%%%%%%%%%%%%%%%%%%%%%%%%%%%%%%%%%%%%%%%%%%%%%%%%%%%%%%%%%%%%%%%%%%%%%%%%%%%%%%%%%%%%%%%%%%%%%
% Custom Commands %%%%%%%%%%%%%%%%%%%%%%%%%%%%%%%%%%%%%%%%%%%%%%%%%%%%%%%%%%%%%%%%%%%%%%%%%%%%%%%%%%%%%%%%%%%%%%%%%%%%%%
%%%%%%%%%%%%%%%%%%%%%%%%%%%%%%%%%%%%%%%%%%%%%%%%%%%%%%%%%%%%%%%%%%%%%%%%%%%%%%%%%%%%%%%%%%%%%%%%%%%%%%%%%%%%%%%%%%%%%%%%

% Correct
\definecolor{correct}{HTML}{009900}
\newcommand\correct[2]{\ensuremath{\:}{\color{red}{#1}}\ensuremath{\to }{\color{correct}{#2}}\ensuremath{\:}}
\newcommand\green[1]{{\color{correct}{#1}}}

% Temporary text markers
\newcommand{\solution}{\noindent\textbf{\large Solution}}       % Alias for the Solution section header
\newcommand{\missing}{\textcolor{red}{\textbf{VALUE MISSING}}}  % Creates a big red "Missing Value" string.
\newcommand{\inprogress}{\textcolor{red}{\textbf{IN PROGRESS}}} % Creates a big red "In Progress" string.
\newcommand{\incomplete}{\textcolor{red}{\textbf{INCOMPLETE}}}  % Creates a big red "Incomplete" string.
\newcommand{\bigred}[1]{\textcolor{red}{\textbf{#1}}}           % Creates a big red string.

% hide parts
\newcommand\hide[1]{}

% Horizontal rule
\newcommand\hr{
    \noindent\rule[0.5ex]{\linewidth}{0.5pt}
}

% Images
\newcommand{\simpleimage}[2][1]{
    \includegraphics[width=\textwidth*{#1}]{#2}
}
\newcommand{\imagecenter}[2][1]{
    \begin{center}
        \includegraphics[width=\textwidth*{#1}]{#2}
    \end{center}
}
\newcommand{\imageleft}[2][1]{
    \begin{flushleft}
        \includegraphics[width=\textwidth*{#1}]{#2}
    \end{flushleft}
}
\newcommand{\imageright}[2][1]{
    \begin{flushright}
        \includegraphics[width=\textwidth*{#1}]{#2}
    \end{flushright}
}

% Figures
\newcommand{\figurecenter}[4]{
    \begin{figure}[H]
        \centering
        \begin{minipage}[H]{#1\textwidth}
            \centering
            \includegraphics[width=\textwidth]{#2}
            \caption{#3}
            \label{#4}
        \end{minipage}
    \end{figure}
}
\newcommand{\figuredual}[6]{
    \begin{figure}[H]
        \begin{minipage}[H]{0.48\textwidth}
            \centering
            \includegraphics[width=\textwidth]{#1}
            \raggedleft
            \caption{#2}
            \label{#3}
        \end{minipage}
        \hfill
        \begin{minipage}[H]{0.48\textwidth}
            \centering
            \includegraphics[width=\textwidth]{#4}
            \caption{#5}
            \label{#6}
        \end{minipage}
    \end{figure}
}
\newcommand{\figuredualsingle}[6]{
    \begin{figure}[H]
        \begin{minipage}[H]{0.48\textwidth}
            \centering
            \includegraphics[width=\textwidth]{#1}
            \subcaption{#2}\label{#6-a}
        \end{minipage}
        \hfill
        \begin{minipage}[H]{0.48\textwidth}
            \centering
            \includegraphics[width=\textwidth]{#3}
            \subcaption{#4}\label{#6-b}
        \end{minipage}
        \caption{#5}\label{#6}
    \end{figure}
}

% Custom font (requires XeLaTeX)
\newcommand{\customfont}[1]{
    \usepackage{mathspec}    % Enable the power of CUSTOM FONTS with XeLaTeX
    \setallmainfonts(Digits,Latin,Greek)[Numbers={Lining,Proportional}]{#1}
    \setallmonofonts[Numbers={Lining,Proportional}]{#1}
    \setallsansfonts[Numbers={Lining,Proportional}]{#1}
    \setmathsfont(Digits,Latin,Greek)[Numbers={Lining,Proportional}]{#1}
    \setmathsf[Numbers={Lining,Proportional}]{#1}
    \setmathbb[Numbers={Lining,Proportional}]{#1}
    \setmathcal[Numbers={Lining,Proportional}]{#1}
}

% Sections alignment
% Gets label of current section
\makeatletter\newcommand{\currentlabel}{\@currentlabelname}\makeatother
% Sets alignment of sections (center/left/right)
\newcommand{\formatsection}[1]{
    \ifthenelse{\equal{#1}{center}}{
        %Makes sections centered
        \titleformat{\section}{\normalfont\Large\bfseries\centering}{\thesection}{1em}{}
    }{
        \ifthenelse{\equal{#1}{right}}{
            %Makes sections right-aligned
            \titleformat{\section}{\normalfont\Large\bfseries\raggedleft}{\thesection}{1em}{}
        }{
            \ifthenelse{\equal{#1}{left}}{
                %Makes sections left-aligned
                \titleformat{\section}{\normalfont\Large\bfseries\raggedright}{\thesection}{1em}{}
            }{
            }
        }
    }
}

% Paragraphs alignment
% Makes the paragraph command appear like a subsubsubsection
\newcommand{\centerparagraphs}{
    \titleformat{\paragraph}{\normalfont\normalsize\bfseries}{\theparagraph}{1em}{}
    \titlespacing*{\paragraph}{0pt}{3.25ex plus 1ex minus .2ex}{1.5ex plus .2ex}
}

%%%%%%%%%%%%%%%%%%%%%%%%%%%%%%%%%%%%%%%%%%%%%%%%%%%%%%%%%%%%%%%%%%%%%%%%%%%%%%%%%%%%%%%%%%%%%%%%%%%%%%%%%%%%%%%%%%%%%%%%
% Environments %%%%%%%%%%%%%%%%%%%%%%%%%%%%%%%%%%%%%%%%%%%%%%%%%%%%%%%%%%%%%%%%%%%%%%%%%%%%%%%%%%%%%%%%%%%%%%%%%%%%%%%%%
%%%%%%%%%%%%%%%%%%%%%%%%%%%%%%%%%%%%%%%%%%%%%%%%%%%%%%%%%%%%%%%%%%%%%%%%%%%%%%%%%%%%%%%%%%%%%%%%%%%%%%%%%%%%%%%%%%%%%%%%

% Augmented matrix
\newenvironment{amatrix}[1]{%
  \left[\begin{array}{@{}*{#1}{c}|c@{}}
}{%
  \end{array}\right]
}

% Colored boxes
\tcbuselibrary{breakable}

\newenvironment{improvement}{\begin{tcolorbox}[
    arc=0mm,
    colback=white,
    colframe=green!60!black,
    title=Improvement,
    fonttitle=\sffamily,
    breakable
]}{\end{tcolorbox}}

\newenvironment{remarkbox}[1]{\begin{tcolorbox}[
    arc=0mm,
    colback=white,
    colframe=white!60!black,
    title=#1,
    fonttitle=\sffamily,
    breakable
]}{\end{tcolorbox}}

% Part
\newcounter{partCounter}
% \part creates a numbered subsection.
\renewcommand{\part}{\subsection{Part (\alph{partCounter})}\stepcounter{partCounter}}

% Sect
% Will write "continued" when content spans >1 page. Argument adjusts the name of the section.
\newenvironment{sect}[1]{
    \pagebreak
    \section{#1}
    \setcounter{partCounter}{1}
    \nobreak\extramarks{}{}\nobreak{}
    \nobreak\extramarks{Section \arabic{section}\ (continued)}{Continued on next page\ldots}\nobreak{}
}{
    \nobreak\extramarks{Section \arabic{section}\ (continued)}{Continued on next page\ldots}\nobreak{}
    \nobreak\extramarks{}{}\nobreak{}
}

%%%%%%%%%%%%%%%%%%%%%%%%%%%%%%%%%%%%%%%%%%%%%%%%%%%%%%%%%%%%%%%%%%%%%%%%%%%%%%%%%%%%%%%%%%%%%%%%%%%%%%%%%%%%%%%%%%%%%%%%
% Theorems %%%%%%%%%%%%%%%%%%%%%%%%%%%%%%%%%%%%%%%%%%%%%%%%%%%%%%%%%%%%%%%%%%%%%%%%%%%%%%%%%%%%%%%%%%%%%%%%%%%%%%%%%%%%%
%%%%%%%%%%%%%%%%%%%%%%%%%%%%%%%%%%%%%%%%%%%%%%%%%%%%%%%%%%%%%%%%%%%%%%%%%%%%%%%%%%%%%%%%%%%%%%%%%%%%%%%%%%%%%%%%%%%%%%%%

% Setting theorem styles
\mdfsetup{skipabove=1em,skipbelow=0em}
\theoremstyle{definition}
\declaretheoremstyle[
    headfont=\bfseries\sffamily\color{ForestGreen!70!black}, bodyfont=\normalfont,
    mdframed={
        linewidth=2pt,
        rightline=false, topline=false, bottomline=false,
        linecolor=ForestGreen, backgroundcolor=ForestGreen!5,
    }
]{thmgreenbox}

\declaretheoremstyle[
    headfont=\bfseries\sffamily\color{NavyBlue!70!black}, bodyfont=\normalfont,
    mdframed={
        linewidth=2pt,
        rightline=false, topline=false, bottomline=false,
        linecolor=NavyBlue, backgroundcolor=NavyBlue!5,
    }
]{thmbluebox}

\declaretheoremstyle[
    headfont=\bfseries\sffamily\color{NavyBlue!70!black}, bodyfont=\normalfont,
    mdframed={
        linewidth=2pt,
        rightline=false, topline=false, bottomline=false,
        linecolor=NavyBlue
    }
]{thmblueline}

\declaretheoremstyle[
    headfont=\bfseries\sffamily\color{RawSienna!70!black}, bodyfont=\normalfont,
    mdframed={
        linewidth=2pt,
        rightline=false, topline=false, bottomline=false,
        linecolor=RawSienna, backgroundcolor=RawSienna!5,
    }
]{thmredbox}

\declaretheoremstyle[
    headfont=\bfseries\sffamily\color{RawSienna!70!black}, bodyfont=\normalfont,
    numbered=no,
    mdframed={
        linewidth=2pt,
        rightline=false, topline=false, bottomline=false,
        linecolor=RawSienna, backgroundcolor=RawSienna!1,
    },
    qed=\qedsymbol
]{thmsandbox}

\declaretheoremstyle[
    headfont=\bfseries\sffamily\color{NavyBlue!70!black}, bodyfont=\normalfont,
    numbered=no,
    mdframed={
        linewidth=2pt,
        rightline=false, topline=false, bottomline=false,
        linecolor=NavyBlue, backgroundcolor=NavyBlue!1,
    },
]{thmlightbluebox}


\declaretheorem[style=thmgreenbox, name=Definition]{definition}
\declaretheorem[style=thmbluebox, numbered=no, name=Example]{example}
\declaretheorem[style=thmbluebox, numbered=no, name=Problem]{problem}
\declaretheorem[style=thmredbox, name=Proposition]{proposition}
\declaretheorem[style=thmredbox, name=Theorem]{theorem}
\declaretheorem[style=thmredbox, name=Lemma]{lemma}
\declaretheorem[style=thmredbox, numbered=no, name=Corollary]{corollary}

\declaretheorem[style=thmsandbox, name=Proof]{replacementproof}
\renewenvironment{proof}[1][\proofname]{\vspace{-10pt}\begin{replacementproof}}{\end{replacementproof}}
\declaretheorem[style=thmlightbluebox, name=Explanation]{tmpexplanation}
\newenvironment{explanation}[1][]{\vspace{-10pt}\begin{tmpexplanation}}{\end{tmpexplanation}}

\declaretheorem[style=thmblueline, numbered=no, name=Remark]{remark}
\declaretheorem[style=thmblueline, numbered=no, name=Claim]{claim}
\declaretheorem[style=thmblueline, numbered=no, name=Note]{note}

\newtheorem*{notation}{Notation}
\newtheorem*{previouslyseen}{As previously seen}
% \newtheorem*{problem}{Problem}
\newtheorem*{observe}{Observe}
\newtheorem*{property}{Property}
\newtheorem*{intuition}{Intuition}

\makeatletter
\newcommand{\exercise}[1]{%
    \def\@exercise{#1}%
    \subsection*{Exercise #1}
}
\newcommand{\subexercise}[1]{%
    \subsubsection*{Exercise \@exercise.#1}
}
\makeatother

% Change QED symbol
\renewcommand\qedsymbol{QED}

% % Plain style (numbered)
% \theoremstyle{plain}
% \newtheorem{definition}{Definition}[subsection]
% \newtheorem{theorem}{Theorem}[subsection]
% \newtheorem{example}{Example}[theorem]
% \newtheorem{corollary}{Corollary}[theorem]
% \newtheorem{lemma}{Lemma}[theorem]
% \newtheorem{proposition}{Proposition}[theorem]
% % Plain style (unnumbered)
% \newtheorem*{definition*}{Definition}
% \newtheorem*{theorem*}{Theorem}
% \newtheorem*{example*}{Example}
% \newtheorem*{corollary*}{Corollary}
% \newtheorem*{lemma*}{Lemma}
% \newtheorem*{proposition*}{Proposition}

% % Definition style (numbered)
% \theoremstyle{definition}
% \newtheorem{exercise}{Exercise}[subsection]
% % Definition style (unnumbered)
% \newtheorem*{exercise*}{Exercise}

% % Remark style (numbered)
% \theoremstyle{remark}
% \newtheorem{remark}{Remark}
% \newtheorem{case}{Case}
% \newtheorem{claim}{Claim}[subsection]
% % Remark style (unnumbered)
% \newtheorem*{remark*}{Remark}
% \newtheorem*{case*}{Case}
% \newtheorem*{claim*}{Claim}

%%%%%%%%%%%%%%%%%%%%%%%%%%%%%%%%%%%%%%%%%%%%%%%%%%%%%%%%%%%%%%%%%%%%%%%%%%%%%%%%%%%%%%%%%%%%%%%%%%%%%%%%%%%%%%%%%%%%%%%%
% Package Settings %%%%%%%%%%%%%%%%%%%%%%%%%%%%%%%%%%%%%%%%%%%%%%%%%%%%%%%%%%%%%%%%%%%%%%%%%%%%%%%%%%%%%%%%%%%%%%%%%%%%%
%%%%%%%%%%%%%%%%%%%%%%%%%%%%%%%%%%%%%%%%%%%%%%%%%%%%%%%%%%%%%%%%%%%%%%%%%%%%%%%%%%%%%%%%%%%%%%%%%%%%%%%%%%%%%%%%%%%%%%%%

% Tikz
\usetikzlibrary{cd, intersections, angles, quotes, calc, positioning, arrows.meta}
\tikzset{
    force/.style={thick, {Circle[length=2pt]}-stealth, shorten <=-1pt}
}

% pgfplots
\pgfplotsset{compat=1.13}

% Enables new list environment that supports nesting
\newlist{legal}{enumerate}{10}
\setlist[legal]{label*=.\arabic*}
\setlist[legal,1]{label=\arabic*}

% TOC & Sections
% Sets TOC depth and change changes its display name
\setcounter{secnumdepth}{3} % Change to 0 for unnumbered sections
\addto\captionsenglish{
  \renewcommand{\contentsname}{Table of Contents}
}

% tabularx
\newcolumntype{Y}{>{\centering\arraybackslash}X}
\newcolumntype{B}[1]{r*{#1}{@{\,}r}}

% Caption/Ref settings
\captionsetup{
    format = hang,
    labelfont = sc,
    skip = 6pt,
    hypcapspace = 96pt
}

% BibLaTex
% Set \cite to \autocite
\let\oldcite\cite
\renewcommand{\cite}{\autocite}
% Makes number created by \supercite to be between brackets.
\let\oldsupercite\supercite
\renewcommand{\supercite}[1]{\textsuperscript{[}\oldsupercite{#1}\textsuperscript{]}}
\DeclareCiteCommand{\supercite}[\mkbibsuperscript]{%
    \iffieldundef{prenote}{}{\BibliographyWarning{Ignoring prenote argument}}%
    \iffieldundef{postnote}{}{\BibliographyWarning{Ignoring postnote argument}}%
}
{\usebibmacro{citeindex}\,\bibopenbracket\usebibmacro{cite}\bibclosebracket}
{\supercitedelim}{}

% PDF export settings
\hypersetup{
    pdftex,
    pdfauthor = {Etienne Collin},
    bookmarksnumbered = true,
    bookmarksopen = true,
    bookmarksopenlevel = 1,
    pdfstartview = Fit,
    pdfpagemode = UseOutlines,
    pdfpagelayout = TwoPageRight,
    colorlinks = true,        % Colors links instead of ugly boxes
    urlcolor = DarkBlue,    % Color for external hyperlinks
    linkcolor = black,        % Color of internal links
    citecolor = black,        % Color of citations
    filecolor = black,        % Color of file links
}

% %http://tex.stackexchange.com/questions/76273/multiple-pdfs-with-page-group-included-in-a-single-page-warning
\pdfsuppresswarningpagegroup=1


% lstlistings setup
\definecolor{myorange}{RGB}{163,21,21}
\definecolor{myteal}{RGB}{43,145,175}
\definecolor{mypurple}{RGB}{198,115,255}
\definecolor{mydarkpurple}{RGB}{111,0,138}
\definecolor{mygreen}{RGB}{0,128,0}

\lstset{
    emphstyle=\bfseries\color{green},
    backgroundcolor=\color{white},                    % Requires xcolor
    commentstyle=\color{mygreen},
    identifierstyle=\color{black},
    keywordstyle=\color{myteal},
    keywordstyle={[2]\color{mydarkpurple}},         % Define these or use a custom language
    stringstyle=\color{myorange}\ttfamily,          % string literal style
    numberstyle=\color{black},                      % the style that is used for the line-numbers
    rulecolor=\color{black},                        % if not set, the frame-color may be changed on line-breaks within not-black text (e.g. comments (green here))
    otherkeywords={self},
    basicstyle=\singlespacing\scriptsize\ttfamily,  % the size of the fonts that are used for the code
    captionpos=b,                                   % sets the caption-position to bottom
    breakatwhitespace=true,                         % sets if automatic breaks should only happen at whitespace
    breaklines=true,                                % sets automatic line breaking
    deletekeywords={},                              % if you want to delete keywords from the given language
    escapeinside={\%*}{*)},                         % if you want to add LaTeX within your code
    extendedchars=true,                             % lets you use non-ASCII characters; for 8-bits encodings only, does not work with UTF-8
    firstnumber=0,                                  % start line enumeration with line 0
    frame=tb,                                       % adds a frame around the code
    keepspaces=true,                                % keeps spaces in text, useful for keeping indentation of code (possibly needs columns=flexible)
    morekeywords={*},                               % if you want to add more keywords to the set
    numbers=left,                                   % where to put the line-numbers; possible values are (none, left, right)
    numbersep=6pt,                                  % how far the line-numbers are from the code
    showspaces=false,                               % show spaces everywhere adding particular underscores; it overrides 'showstringspaces'
    showstringspaces=false,                         % underline spaces within strings only
    showtabs=false,                                 % show tabs within strings adding particular underscores
    stepnumber=1,                                   % the step between two line-numbers. If it's 1, each line will be numbered
    tabsize=4,                                      % sets default tabsize to 4 spaces
    title=\lstname,                                 % show the filename of files included with \lstinputlisting; also try caption instead of title
    belowskip=-1\baselineskip,                      % sets spacing under listing
    % belowcaptionskip=1\baselineskip,              % sets spacing under caption
    % literate=%
    %     {0}{{\textcolor{yellow}{0}}}{1}%
    %     {1}{{\textcolor{yellow}{1}}}{1}%
    %     {2}{{\textcolor{yellow}{2}}}{1}%
    %     {3}{{\textcolor{yellow}{3}}}{1}%
    %     {4}{{\textcolor{yellow}{4}}}{1}%
    %     {5}{{\textcolor{yellow}{5}}}{1}%
    %     {6}{{\textcolor{yellow}{6}}}{1}%
    %     {7}{{\textcolor{yellow}{7}}}{1}%
    %     {8}{{\textcolor{yellow}{8}}}{1}%
    %     {9}{{\textcolor{yellow}{9}}}{1}%
    %     {.0}{{\textcolor{yellow}{.0}}}{2}% Following is to ensure that only periods
    %     {.1}{{\textcolor{yellow}{.1}}}{2}% followed by a digit are changed.
    %     {.2}{{\textcolor{yellow}{.2}}}{2}%
    %     {.3}{{\textcolor{yellow}{.3}}}{2}%
    %     {.4}{{\textcolor{yellow}{.4}}}{2}%
    %     {.5}{{\textcolor{yellow}{.5}}}{2}%
    %     {.6}{{\textcolor{yellow}{.6}}}{2}%
    %     {.7}{{\textcolor{yellow}{.7}}}{2}%
    %     {.8}{{\textcolor{yellow}{.8}}}{2}%
    %     {.9}{{\textcolor{yellow}{.9}}}{2}%
}

%%%%%%%%%%%%%%%%%%%%%%%%%%%%%%%%%%%%%%%%%%%%%%%%%%%%%%%%%%%%%%%%%%%%%%%%%%%%%%%%%%%%%%%%%%%%%%%%%%%%%%%%%%%%%%%%%%%%%%%%
% End of Environment Setup %%%%%%%%%%%%%%%%%%%%%%%%%%%%%%%%%%%%%%%%%%%%%%%%%%%%%%%%%%%%%%%%%%%%%%%%%%%%%%%%%%%%%%%%%%%%%
%%%%%%%%%%%%%%%%%%%%%%%%%%%%%%%%%%%%%%%%%%%%%%%%%%%%%%%%%%%%%%%%%%%%%%%%%%%%%%%%%%%%%%%%%%%%%%%%%%%%%%%%%%%%%%%%%%%%%%%%